\documentclass{exam}
\usepackage{amsthm,amsmath,amssymb,fullpage}
\usepackage{marvosym}
\usepackage{graphicx}

\DeclareMathAlphabet{\mymathbb}{U}{BOONDOX-ds}{m}{n}

\begin{document}
\noindent \textbf{Name: Adam Frank}\hfill \textbf{\today}

\vspace{.3cm}
\hrule
\begin{center}
{\bf \Large{Math 637: Homework Chapter 3}}
\end{center}
\hrule
\vspace{.3cm}


\begin{questions}
\question(3.1.1)\\
Let $\varphi: G\rightarrow H$ be a homomorphism and let $E$ be a subgroup of $H$.  Prove that $\varphi^{-1}(E)\leq G$.  If $E\trianglelefteq H$ prove that $\varphi^{-1}(E)\trianglelefteq G$.  Deduce that $\ker \varphi \trianglelefteq G$.

\begin{proof}
  Let $a,b\in\varphi^{-1}(E)$.  Then $\varphi(ab^{-1}) = \varphi(a)(\varphi(b))^{-1}$.  Since $\varphi(a),\varphi(b)\in E$ and $E$ is a subgroup, then $\varphi(a)(\varphi(b))^{-1}\in E$.  Hence $\varphi^{-1}(E)\leq G$.

  Suppose $E\trianglelefteq H$ and let $g\in G$.  Let $x\in \varphi^{-1}(E)$ so that $\varphi(x)\in E$. We will see that $gxg^{-1}\in\varphi^{-1}(E)$ which is the same as $\varphi(gxg^{-1})\in E$.  From this, by Theorem 6 part (5) of chapter 3.1, we will have that $\varphi(E)\trianglelefteq G$.  Note that $\varphi(gxg^{-1}) = \varphi(g)\varphi(x)(\varphi(g))^{-1}$ is $\varphi(x)$ conjugate with $\varphi(g)$.  By $E\trianglelefteq H$, we have $\varphi(g)\varphi(x)(\varphi(g))^{-1} = \varphi(gxg^{-1})\in E$.

  To show that $\ker \varphi \trianglelefteq G$ set $E=\{\mymathbb 1\}$.  This is always normal in any group since $g\{\mymathbb 1\}g^{-1} = \{gg^{1}\}=\{\mymathbb 1\}$.  But then $\ker\varphi = \varphi^{-1}(E)$ in this case, so the above entails that this is normal in $G$.
\end{proof}

\question(3.1.3)\\
Let $A$ be an abelian group and $B\leq A$.  Prove that $A/B$ is abelian.  Give an example of a non-abelian group $G$ containing a proper normal subgroup $N$ such that $G/N$ is abelian.

\begin{proof}
  First we show that $A/B$ is a group.  As noted in the examples after the definition of the natural projection, any abelian group is normal.  From the theorem that all normal subgroups are the kernel of some homomorphism, we then know that $B = \ker \pi$ for some homomorphism $\pi$.  Now from the theorem that the quotient group is defined for kernels of homomorphisms, we have that $A/B$ is a group.

  Now let $x,y\in A$ so that $xB,yB\in A/B$.  Then

  \begin{align*}
      (xB)(yB) = (xy)B = (yx)B = (yB)(xB)
  \end{align*}

  The first and last equality come from the definition of the quotient group operation.  The middle equality is by the abelianness of $A$.

  \vspace{1cm}

  Here is an example of a non-abelian group with a proper normal subgroup such that the quotient is abelian:  Set $A=D_4$ and $B=\langle r\rangle$, which is to say that $B$ is the group of rotations.  Then to see that $B\trianglelefteq D_4$ observe that for any $i=0,1,2,3$ we have $r^iBr^{-i} = B$ just because rotations commute with each other.  If $r^is$ is any reflection, and $r^j\in B$ then we'll show that $(r^is)r^j(r^is)^{-1}\in B$, which as we've noted earlier, suffices to show that $B\trianglelefteq A$.

  \begin{align*}
    (r^is)r^j(r^is)^{-1} &= r^isr^jsr^{-i} \\\\
    &= r^ir^{-j}ssr^{-i} \\\\
    &= r^{i-j-i} = r^{-j}\in B
  \end{align*}

  Finally, to see that $A/B$ is abelian it is enough to show that it has order 2.  Since

  \begin{align*}
    sB = s\{\mymathbb 1, r, r^2, r^3\} = \{s,sr,sr^2,sr^3\}
  \end{align*}

  Since there can be no other cosests, the only two elements of $A/B$ are $B$ and $sB$.  Since trivially any group of order 2 is abelian, $A/B$ is abelian.


\end{proof}

\question(3.1.11(a))\\
Let $F$ be a field and $G = \left\{\begin{pmatrix}
  a & b \\ 0 & c
\end{pmatrix}| a,b,c\in F, ac\ne 0\right\} \leq GL_2(F)$.

(a) Prove that the map $\varphi:\begin{pmatrix}
  a & b \\ 0 & c
\end{pmatrix} \mapsto a$ is a surjective homomorphism from $G$ onto $F^\times$.  Describe the fibers and kernel of $\varphi$.

\begin{proof}
  First that it's a homomorphism:

  \begin{align*}
    \varphi\left[
    \begin{pmatrix}
        a & b \\ 0 & c
    \end{pmatrix}
    \begin{pmatrix}
        x & y \\ 0 & z
    \end{pmatrix}\right] &=
    \varphi
    \begin{pmatrix}
        ax & ay+bz \\ 0 & cz
    \end{pmatrix} \\\\
  &= ax \\\\
  &= \left[\varphi
  \begin{pmatrix}
      a & b \\ 0 & c
  \end{pmatrix}\right]
  \left[\varphi
  \begin{pmatrix}
      x & y \\ 0 & z
  \end{pmatrix}\right]
  \end{align*}

  Next we see that it's surjective.  If $a\in F$ then $\varphi
  \begin{pmatrix}
      a & 0 \\ 0 & \mymathbb 1
  \end{pmatrix} = a$.  Note that $\mymathbb 1\in F$ because $F$ is a field.
\end{proof}

\question(3.1.22(a))\\
Prove that if $H$ and $K$ are normal subgroups of a group $G$ then their intersection $H\cap K$ is also a normal subgroup of $G$.

\begin{proof}
  If $g\in G$ then it suffices to show that $g(H\cap K)g^{-1}\subseteq H\cap K$.  So let $gxg^{-1}\in g(H\cap K)g^{-1}$ where $x\in H\cap K$.  Because $H$ is normal in $G$ we have $gxg^{-1}\in H$, and because $K$ is normal in $G$ we have $gxg^{-1}\in K$.  Hence $gxg^{-1}\in H\cap K$.
\end{proof}

\question(3.1.42)\\
Assume both $H$ and $K$ are normal subgroups of $G$ with $H\cap K=\mymathbb 1$.  Prove that $xy=yx$ for all $x\in H$ and $y\in K$.

\begin{proof}
  We first prove $x^{-1}y^{-1}xy \in H$.  We already know $x\in H$ therefore $x^{-1}\in H$.  Because $H\trianglelefteq G$ then $y^{-1}x(y^{-1})^{-1}\in H$ since this is conjugation of $x$ by $y^{-1}$.  Then $x^{-1}y^{-1}xy\in H$.

  Next we show $x^{-1}y^{-1}xy \in K$.  We already know $y\in K$.  Because $K\trianglelefteq G$ then $x^{-1}y(x^{-1})^{-1}\in H$ since this is conjugation of $y$ by $x^{-1}$.  Then $x^{-1}y^{-1}xy\in K$.

  Hence $x^{-1}y^{-1}xy \in H\cap K$ and therefore $x^{-1}y^{-1}xy =\mymathbb 1$.  Multiply by $yx$ and you have

  \begin{align*}
    xy = yx
  \end{align*}

  as desired.
\end{proof}

\question(3.2.4)\\
Show that if $|G|=pq$ for some primes $p$ and $q$ (not necessarily distinct) then either $G$ is abelian or $Z(G)=1$.

\begin{proof}
  If $Z(G)\ne \mymathbb 1$ then the order $|Z(G)|$ can only be either $p$ or $q$ or $pq$.  If the order is $pq$ then $Z(G)=G$ and so $G$ is abelian.  So it suffices to show that the order cannot be either $p$ or $q$.

  Suppose for contradiction that the order of $Z(G)$ is $p$, which is to say $|Z(G)|=p$.  Let $g\in G\smallsetminus Z(G)$ so that $g$ does not commute with some element.  As a preliminary remark note that no element of $\langle g\rangle$ is in $Z(G)$ either, since if $g^i$ commutes with every $x\in G$ then

  \begin{align*}
    g^i x &= xg^i \Leftrightarrow \\\\
    gxg^{i-1} &= xg^i \Leftrightarrow \\\\
    gx &= xg\\
  \end{align*}

  So $g$ commutes with every $x\in G$ contrary to assumption.

  Now the order of $g$ is $p$ or $q$ or $pq$.  If the order is $pq$ then $G$ is cyclic, so abelian, so $|Z(G)|=pq$ contrary to assumption.

  Suppose the order of $g$ is $q$.  For every $z_1,z_2\in Z(G)$ and $h_1,h_2\in \langle g\rangle$, the elements $z_1h_1 = z_2h_2$ if and only if $z_2^{-1}z_1 = h_2h_1^{-1}\in \langle g\rangle \cap Z(G)$.  So in that case $z_1=z_2$ and $h_1=h_2$.  Thus every $z\in Z(G), h\in \langle g\rangle$ corresponds to a unique $zh\in \langle Z(G),g\rangle$.  Thus the group generated by these two sets has size at least $pq$, i.e. $|\langle Z(G),g\rangle|\leq pq$.  Hence $G = \langle Z(G),g\rangle$.  Thus if we pick any $h\in G\smallsetminus Z(G)$ then $h\in \langle g\rangle$.  Now $h$ commutes with everything in $Z(G)$ by definition, and it commutes with everything in $\langle g\rangle$ trivially.  Hence $h\in Z(G)$, a contradiciton.  But then $G\smallsetminus Z(G)= \emptyset$, which contradicts $|Z(G)|=p$.

  Finally suppose that $|g|\ne q$ so that $|g|=p$.
\end{proof}

\question(3.2.7)\\

\begin{proof}
\end{proof}

\question(3.2.12)\\

\begin{proof}
\end{proof}

\question(3.2.16)\\

\begin{proof}
\end{proof}

\question(3.3.3)\\

\begin{proof}
\end{proof}

\question(3.3.8)\\

\begin{proof}
\end{proof}

\question(3.3.9)\\

\begin{proof}
\end{proof}

\question(3.4.1)\\

\begin{proof}
\end{proof}

\question(3.4.11)\\

\begin{proof}
\end{proof}

\question(3.5.12)\\

\begin{proof}
\end{proof}


\end{questions}

\end{document}
